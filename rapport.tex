\documentclass{report}
\usepackage[latin1]{inputenc}
\usepackage[T1]{fontenc}
\usepackage[francais]{babel}
\usepackage{lmodern}
\usepackage{mathrsfs}
\usepackage{mathrsfs}
\usepackage{textcomp}
\usepackage{listings}
\usepackage{siunitx}
\usepackage{tikz}

\usepackage{color}

\definecolor{mygreen}{rgb}{0,0.6,0}
\definecolor{mygray}{rgb}{0.5,0.5,0.5}
\definecolor{mymauve}{rgb}{0.58,0,0.82}

\lstset{ %
  backgroundcolor=\color{white},   % choose the background color; you must add \usepackage{color} or \usepackage{xcolor}; should come as last argument
  basicstyle=\footnotesize,        % the size of the fonts that are used for the code
  breakatwhitespace=false,         % sets if automatic breaks should only happen at whitespace
  breaklines=true,                 % sets automatic line breaking
  captionpos=b,                    % sets the caption-position to bottom
  commentstyle=\color{mygreen},    % comment style
  deletekeywords={...},            % if you want to delete keywords from the given language
  escapeinside={\%*}{*)},          % if you want to add LaTeX within your code
  extendedchars=true,              % lets you use non-ASCII characters; for 8-bits encodings only, does not work with UTF-8
  frame=single,	                   % adds a frame around the code
  keepspaces=true,                 % keeps spaces in text, useful for keeping indentation of code (possibly needs columns=flexible)
  keywordstyle=\color{blue},       % keyword style
  language=C,                      % the language of the code
  morekeywords={*,...},           % if you want to add more keywords to the set
  numbers=left,                    % where to put the line-numbers; possible values are (none, left, right)
  numbersep=5pt,                   % how far the line-numbers are from the code
  numberstyle=\tiny\color{mygray}, % the style that is used for the line-numbers
  rulecolor=\color{black},         % if not set, the frame-color may be changed on line-breaks within not-black text (e.g. comments (green here))
  showspaces=false,                % show spaces everywhere adding particular underscores; it overrides 'showstringspaces'
  showstringspaces=false,          % underline spaces within strings only
  showtabs=false,                  % show tabs within strings adding particular underscores
  stepnumber=2,                    % the step between two line-numbers. If it's 1, each line will be numbered
  stringstyle=\color{mymauve},     % string literal style
  tabsize=2,	                   % sets default tabsize to 2 spaces
  title=\lstname                   % show the filename of files included with \lstinputlisting; also try caption instead of title
}
\oddsidemargin=-0,8cm
\headsep=-1,5cm
\textwidth 18cm
\textheight 25,3cm

\newenvironment{introduction}{%
     \par\noindent\textbf{Remarques}:\noindent
}{
     \par\noindent
}

\author{Grand Maxence G1, Muller Lucie}
\title{Syst\`emes et R\'eseaux : Impl\'ementation d'un pseudo serveur FTP }
\date{06/04/2017}

\begin{document}

\maketitle
\tableofcontents

\chapter*{Introduction}
%Intro projet 
	Durant ce projet, nous avons du impl\'ementer un serveur de fichier, inspir\'e des serveurs FTP. Nous sommes all\'es d'un serveur avec une fonction simple (:la r\'ecuperation de fichiers \`a distance) vers un serveur permettant des actions plus complexes et, de ce fait, plus performant.\\
    \chapter{Serveur FTP concurrent}
      \section{Objectif de la r\'ealisation}
	Dans cette partie, nous avons du impl\'ementer la fonction simple du serveur : la r\'ecuperation de fichier \`a distance en donnant le nom du fichier \'a partir du client. Le serveur cr\'e\'e un nouveau fichier sur lequel, \`a l'aide d'un buffer, il copie le contenu du premier fichier. Ce nouveau fichier est sauvegard\'e a l'emplacement courant du client, sous le nom "nomdupremier fichier1"\\
      \section{Code source comment\'e}
      \subsection{client.c}
      \begin{lstlisting}
int main(int argc, char **argv)
{
  int clientfd, fd;
  char *buf, *host, *hidefile;
  struct timeval start, end;
  size_t len=0, n=0;

  if (argc != 2) {
    fprintf(stderr, "usage: %s <host>\n", argv[0]);
    exit(0);
  }
  host = argv[1];//On recupere le hostname du serveur (127.0.0.1 pour les tests)

  clientfd = Open_clientfd(host, port);//On demande la connexion au serveur

  req = malloc(sizeof(struct Request));//On alloue de la memoire pour la Request
  req->clientfd = clientfd;//On recupere le descripteur du socket

  Rio_readinitb(&rio, clientfd);//On initialise Rio

  printf("client connected to server %s\n", host);

  while (1) {
    printf("Tapez votre requete : \n");
    buf = (char*)malloc(sizeof(char)*MAXLINE);
    Fgets(buf, MAXLINE, stdin);
    req->filename = (char*)malloc(sizeof(char)*MAXLINE);
    req->cmd = (char*)malloc(sizeof(char)*MAXLINE);
    req->content = (char*)malloc(sizeof(char)*MAXLINE);
    sscanf(buf, "%s %s %s", req->cmd, req->filename, req->content);//On recupere la requette tapee par le client

    if(!strcmp(req->cmd, "get")){
      gettimeofday(&start, NULL);
      Rio_writen(req->clientfd, buf, strlen(buf));//On envoie la requete au serveur
      fd = open(strcat(req->filename, "1"), O_RDWR | O_CREAT, 0666);//On creer un fichier pour le téléchargement
      free (buf);
      buf = (char*)malloc(sizeof(char)*MAXSEND);
      while((n=Rio_readn(req->clientfd, buf, MAXSEND)) > 0) {//Tant qu'on recoit des donnees, on les ecrit dans le fichier
        rio_writen(fd, buf, n);
        len+=n;
        free(buf);
        buf = (char*) malloc(sizeof(char)*MAXSEND);
      }
      Close(fd);
      gettimeofday(&end, NULL);
      double temps = ((end.tv_sec+(double)end.tv_usec/1000000)-(start.tv_sec+(double)start.tv_usec/1000000));
      printf("%lu bytes received in %f sec (%f bytes / sec) \n",len, temps, ((double)(len/temps)));//On affiche les statistiques
      if(len == 0){
        printf("Le téléchargement a echoée\n");
        remove(req->filename);
      }
    }else if(!strcmp(req->cmd, "bye")){//On ferme la connexion
      Rio_writen(req->clientfd, req->cmd, strlen(req->cmd));
      printf("fin de la connexion\n");
      exit(0);
    }else{//Si la commande est inconnue on ferme la connexion
      printf("Commande %s inconnue\n", req->cmd);
      printf("fin de la connexion\n");
      exit(0);
    }
  }
}
      \end{lstlisting}
      \subsection{server.c}
      \begin{lstlisting}
      int main(int argc, char **argv)
      {
      	Signal(SIGINT, stop);
      	Signal(SIGCHLD, handler);
          pid_t p;
          int listenfd, connfd;
          socklen_t clientlen;
          struct sockaddr_in clientaddr;
          char client_ip_string[INET_ADDRSTRLEN];
          char client_hostname[MAX_NAME_LEN];

          clientlen = (socklen_t)sizeof(clientaddr);

          listenfd = Open_listenfd(port);
          for(int i=0; i<NB_PROC; i++){
              if((p=Fork())==0){
                  while (1) {
                      if((connfd = Accept(listenfd, (SA *)&clientaddr, &clientlen))>=0){
      									//On accept la connexion entrante

                          /* determine the name of the client */
                          Getnameinfo((SA *) &clientaddr, clientlen,client_hostname, MAX_NAME_LEN, 0, 0, 0);

                          /* determine the textual representation of the client's IP address */
                          Inet_ntop(AF_INET, &clientaddr.sin_addr, client_ip_string,INET_ADDRSTRLEN);


                          printf("server connected to %s (%s) %d\n", client_hostname,client_ip_string, getpid());
                          req = malloc(sizeof(struct Request));
                          req->connfd = connfd;

                          readRequest(req);//On lit la requete du client
      										Close(connfd);
                      }
                  }
          				freeRequest(req);
                  exit(0);
              }
              child[i] = p;
          }

          for(int i=0; i<NB_PROC; i++)
              waitpid(child[i], NULL, 0);
      		//On attend tous les fils

          exit(0);
      }

      void handler(int sig){
      	//On s'occupe des zombis
          waitpid(-1, NULL, WNOHANG|WUNTRACED);
          return;
      }
      void stop(int sig){
      	//Si on recoit un SIGCTR, on tue tous les fils
          for(int i=0; i<NB_PROC; i++){
              kill(SIGKILL, child[i]);
          }
          exit(0);
      }

      void readRequest(struct Request *req){

          size_t n;
          req->cmd = (char*)malloc(sizeof(char)*MAXLINE);
          req->filename = (char*)malloc(sizeof(char)*MAXLINE);
          req->content = (char*)malloc(sizeof(char)*MAXLINE);
          char* request = (char *)malloc(sizeof(char)*MAXLINE);
          rio_t rio;
          Rio_readinitb(&rio, req->connfd);
          n=Rio_readlineb(&rio, request, MAXLINE);
          fflush(stdout);
          request[n-1] = '\0';

          sscanf(request, "%s %s %s", req->cmd, req->filename, req->content);
      		//On recupere la requte , et remplis notre Request
      		stat(req->filename, &req->sbuf);
          if(!strcmp("get", req->cmd)){
          	get(req);
          }else if (!strcmp("bye", req->cmd) ) {
          	printf("fin de la connexion");
      			Close(req->connfd);
          }
          fflush(stdout);
      }

      void get(struct Request *req){
      	char buf[MAXSEND];
      	size_t n,err;
      	rio_t rio;


      		//On envoie d'un coup
      		int srcfd;
          char *srcp;
      		//Rio_readinitb(&rio, fd);
          srcfd = Open(req->filename, O_RDONLY, 0);
          srcp = Mmap(0, req->sbuf.st_size, PROT_READ, MAP_PRIVATE, srcfd, 0);
          Close(srcfd);
          Rio_writen(req->connfd, srcp, req->sbuf.st_size);
          Munmap(srcp, req->sbuf.st_size);
      }
      \end{lstlisting}
      \section{Tests}
\begin{lstlisting}[frame=single,basicstyle=\footnotesize,language=bash]
maxence@Sybil:~$ cat Makefile
CC=gcc
CFLAGS=-Wall
LIBS=-lpthread

all: server client


server: server.o csapp.o tell_wait.o
$(CC) $(CFLAGS) -o server server.o csapp.o tell_wait.o  $(LIBS)

client: client.o csapp.o
$(CC) $(CFLAGS) -o client  client.o csapp.o $(LIBS)

client.o: client.c csapp.c client.h csapp.h
$(CC) $(CFLAGS) -c client.c csapp.c $(LIBS)

server.o: server.c csapp.c server.h csapp.h
		$(CC) $(CFLAGS) -c server.c csapp.c $(LIBS)

tell_wait.o: tell_wait.c csapp.c tell_wait.h csapp.h
	$(CC) $(CFLAGS) -c tell_wait.c csapp.c $(LIBS)

csapp.o: csapp.c csapp.h
		$(CC) $(CFLAGS) -c csapp.c $(LIBS)

clean:
  	rm server client *.o
maxence@Sybil:~$ ls
backupServer  client.c  client.o  csapp.h  Makefile  server    server.h  tell_wait.c  tell_wait.o  test_pipe.c
client        client.h  csapp.c   csapp.o  master.c  server.c  server.o  tell_wait.h  test
maxence@Sybil:~$ ./client 127.0.0.1
client connected to server 127.0.0.1
Tapez votre requete :
get Makefile
617 bytes received in 0.002723 sec (226590.103143 bytes / sec)
Tapez votre requete :
bye
fin de la connexion
maxence@Sybil:~$ cat Makefile1
CC=gcc
CFLAGS=-Wall
LIBS=-lpthread

all: server client


server: server.o csapp.o tell_wait.o
  $(CC) $(CFLAGS) -o server server.o csapp.o tell_wait.o  $(LIBS)

client: client.o csapp.o
	$(CC) $(CFLAGS) -o client  client.o csapp.o $(LIBS)

client.o: client.c csapp.c client.h csapp.h
	$(CC) $(CFLAGS) -c client.c csapp.c $(LIBS)

server.o: server.c csapp.c server.h csapp.h
	$(CC) $(CFLAGS) -c server.c csapp.c $(LIBS)

tell_wait.o: tell_wait.c csapp.c tell_wait.h csapp.h
	$(CC) $(CFLAGS) -c tell_wait.c csapp.c $(LIBS)

csapp.o: csapp.c csapp.h
	$(CC) $(CFLAGS) -c csapp.c $(LIBS)

clean:
	rm server client *.o
\end{lstlisting}

    \chapter{D\'ecoupage du fichier}
      \section{Objectif de la r\'ealisation}
	Dans cette partie, nous avons du ameliorer le serveur afin de ne pas envoyer le fichier en une seule fois, mais en plusieurs paquets. Cela permet de ne pas surcharger la m\'emoire si un fichier de grande taille est transfer\'e. Pour cela on utilise une boucle qui envoie a chaque fois MAXSEND (maximum d'octets envoy\'es) octets au client.\\
      \section{Code source comment\'e}
\subsection{server.c}
Seul la fonction get de server.c a \'et\'e modifi\'ee

  \begin{lstlisting}
void get(struct equest *req){
  	char buf[MAXSEND];
  	size_t n,err;
  	rio_t rio;

  	int fd = open(req->filename, O_RDONLY);
  	Rio_readinitb(&rio, fd);
  	/*
  Tant qu'on a ecris, on continue d'ecrire
  	*/
  	while ((n = Rio_readlineb(&rio, buf, MAXSEND)) > 0) {
  			err = rio_writen(req->connfd, buf, n);
  	    if(err == -1){//On stoppe si le client s'arrete
  		     	fprintf(stderr, "Arret innatendu du client\n");
  		     	break;
  	    }
  	}

  }
  \end{lstlisting}
      \section{Tests}
\begin{lstlisting}[frame=single,basicstyle=\footnotesize,language=bash]
maxence@Sybil:~$ cat test
backupServer
client
client.c
client.h
client.o
csapp.c
csapp.h
csapp.o
Makefile
Makefile1
master.c
server
server.c
server.h
server.o
tell_wait.c
tell_wait.h
tell_wait.o
test
test_pipe.c
maxence@Sybil:~$ ./client 127.0.0.1
client connected to server 127.0.0.1
Tapez votre requete :
get test
186 bytes received in 0.009943 sec (18706.611932 bytes / sec)
Tapez votre requete :
bye
fin de la connexion
maxence@Sybil:~$ cat test1
backupServer
client
client.c
client.h
client.o
csapp.c
csapp.h
csapp.o
Makefile
Makefile1
master.c
server
server.c
server.h
server.o
tell_wait.c
tell_wait.h
tell_wait.o
test
test_pipe.c
\end{lstlisting}


    \chapter{Gestion simple des pannes cot\'e client}
      \section{Objectif de la r\'ealisation}
	Dans cette partie, nous devons g\`erer les pannes du client, c'est-a-dire le comportement du serveur au cas o\`u le client s'arr\^eterais durant le transfert. Un fichier temporaire est cr\'e\'e au début du processus et lorsque ce dernier est termin\'e le fichier est supprim\'e. Il n'est pas supprim\'e si le processus n'a pas finit. Le client v\'erifie, a l'appel du transfert, si un tel fichier temporaire existe. Si c'est le cas cela signifie que le serveur a d\'ej\`a commenc\'e un transfert mais que celui ci a \'etait arr\^et\'e avant sa terminaison. Le client r\'ecup\`e alors ce fichier temporaire et sait quelles donn\'ees manquantes il lui reste a t\'el\'echarger. \\
      \section{Code source comment\'e}
      Seul le fichier server.c a \'et\'e modifi\'e
            \subsection{client.c}
            \begin{lstlisting}

if(!strcmp(req->cmd, "get")){
  gettimeofday(&start, NULL);
  Rio_writen(req->clientfd, buf, strlen(buf));//On envoie la requete au serveur
  hidefile = (char*)malloc(sizeof(char)*MAX_NAME_LEN);
  hidefile[0]='T';
  strcat(hidefile, req->filename);//On creer le nom du fichier temporaire
  if(stat(hidefile,&req->sbuf) != 0){//Si le fichier temporaire n'existe pas
    open(hidefile, O_RDWR | O_CREAT, 0666);//On le cree
    fd = open(strcat(req->filename, "1"), O_RDWR | O_CREAT, 0666);//On cree un fichier pour le téléchargement
    free (buf);
    buf = (char*)malloc(sizeof(char)*MAXSEND);
    while((n=Rio_readn(req->clientfd, buf, MAXSEND)) > 0) {//Tant qu'on recoit des donnees, on les ecrit dans le fichier
      rio_writen(fd, buf, n);
      len+=n;
      free(buf);
      buf = (char*) malloc(sizeof(char)*MAXSEND);
    }
    Close(fd);
    gettimeofday(&end, NULL);
    double temps = ((end.tv_sec+(double)end.tv_usec/1000000)-(start.tv_sec+(double)start.tv_usec/1000000));
    printf("%lu bytes received in %f sec (%f bytes / sec) \n",len, temps, ((double)(len/temps)));//On affiche les statistiques
      if(len == 0){
      printf("Le téléchargement a echoue\n");
      remove(req->filename);
    }
    remove(hidefile);
  }else{//Si le fichier temporaire existe
    fd = open(strcat(req->filename, "1"), O_WRONLY);
    stat(req->filename,&req->sbuf);
    int dejaLu = req->sbuf.st_size;//On recupere la taille des donnees deja telechargees
    free (buf);
    buf = (char*)malloc(sizeof(char)*MAXSEND);
    while((n=Rio_readn(req->clientfd, buf, MAXSEND)) > 0) {
      int tmp = dejaLu;
      dejaLu-=n;
      if(dejaLu <= 0){//On ecrit si ce sont des donnees non téléchargées
      rio_writen(fd, buf, n);
      free(buf);
      buf = (char*) malloc(sizeof(char)*MAXSEND);
      len+=n;
    }else{//Sinon on déplace le curseur
      if(n<tmp)
          lseek(fd, n, SEEK_CUR);
      else
          lseek(fd, n-tmp, SEEK_CUR);
      }
    }
    gettimeofday(&end, NULL);
    double temps = ((end.tv_sec+(double)end.tv_usec/1000000)-(start.tv_sec+(double)start.tv_usec/1000000));
      printf("%lu bytes received in %f sec (%f bytes / sec) \n",len, temps, ((double)(len/temps)));
      remove(hidefile);
    }
  }else if(!strcmp(req->cmd, "bye")){//On ferme la connexion
    Rio_writen(req->clientfd, req->cmd, strlen(req->cmd));
    printf("fin de la connexion\n");
    exit(0);
  }else{//On stop la connexion
    printf("Commande %s inconnue\n", req->cmd);
    printf("fin de la connexion\n");
    exit(0);
  }
}
            \end{lstlisting}

      \section{Tests}
      \begin{lstlisting}[frame=single,basicstyle=\footnotesize,language=bash]
maxence@Sybil:~$ rm test1
maxence@Sybil:~$ head -3 test > test1
maxence@Sybil:~$ touch Ttest
maxence@Sybil:~$ ./client 127.0.0.1
client connected to server 127.0.0.1
Tapez votre requete :
get test
186 bytes received in 0.000816 sec (227977.949737 bytes / sec)
Tapez votre requete :
bye
fin de la connexion
maxence@Sybil:~$ cat test1
backupServer
client
client.c
client.h
client.o
csapp.c
csapp.h
csapp.o
Makefile
Makefile1
master.c
server
server.c
server.h
server.o
tell_wait.c
tell_wait.h
tell_wait.o
test
\end{lstlisting}

    \chapter{Serveur FTP avec \'equilibrage des charges}
      \section{Objectif de la r\'ealisation}
	Dans cette partie, nous avons ajout\'e un repetisseur de charge. Il sert a partager les differentes requ\^etes des clients entre plusieurs serveurs dis "esclaves". Cette repartission est g\`er\'ee par un serveur "maitre" qui re\c coit toutes les requ\^etes avant de les envoyer aux esclaves.
      \section{Code source comment\'e}
      \subsection{master.c}
      \begin{lstlisting}
int sCourant = 0;

int slave[NB_PROC];

int next(){//Renvoie le prochain esclave
	int p = slave[sCourant];
	sCourant = ++sCourant % NB_PROC;
	return p;
}
int main(int argc, char **argv)
{
  int listenfd, connfd;
  socklen_t clientlen;
  struct sockaddr_in clientaddr;
  char client_ip_string[INET_ADDRSTRLEN];
  char client_hostname[MAX_NAME_LEN];
	rio_t rio;

  clientlen = (socklen_t)sizeof(clientaddr);

  listenfd = Open_listenfd(port);

  for(int i =0; i<NB_PROC; i++){
    slave[i] = port+i+1;
  }
  while(1){
    if((connfd = Accept(listenfd, (SA *)&clientaddr, &clientlen))>=0){

      /* determine the name of the client */
      Getnameinfo((SA *) &clientaddr, clientlen,client_hostname, MAX_NAME_LEN, 0, 0, 0);

      /* determine the textual representation of the client's IP address */
      Inet_ntop(AF_INET, &clientaddr.sin_addr, client_ip_string,INET_ADDRSTRLEN);


      printf("server connected to %s (%s) %d\n", client_hostname,client_ip_string, getpid());

  		Rio_readinitb(&rio, connfd);
  		int p = next();
  		Rio_writen(connfd, &p, sizeof(p));//Envoie au client le numero de port de l'esclve traitant

      Close(connfd);//Ferme la connexion avec le client
    }
  }
  exit(0);
}
      \end{lstlisting}
      \subsection{server.c}
      \begin{lstlisting}
clientfd = Open_clientfd(host, port);//On demande la connexion au serveur maitre
printf("client connected to server %s\n", host);

Rio_readn(clientfd, &port2, sizeof(int)); //On recupert le port du serveur esclave

printf("Vous aller être redirigé vers l'esclave %d\n", port2);

clientfd = Open_clientfd(host, port2);//On demande la connexion au serveur esclave

req = malloc(sizeof(struct Request));//On alloue de la memoire pour la Request
req->clientfd = clientfd;//On recupere le descripteur du socket
      \end{lstlisting}
      \section{Tests}
\begin{lstlisting}[frame=single,basicstyle=\footnotesize,language=bash]
maxence@Sybil:~$ ls > test
maxence@Sybil:~$ ./client 127.0.0.1
client connected to server 127.0.0.1
Vous aller être redirigé vers l'esclave 2122
Tapez votre requete :
get test
238 bytes received in 0.004584 sec (51921.582857 bytes / sec)
Tapez votre requete :
bye
fin de la connexion
maxence@Sybil:~$ cat test
backupServer
client
client.c
client.h
client.o
csapp.c
csapp.h
csapp.o
debug
Makefile
master
master.c
master.o
projetFTPBackup
rapport.pdf
rapport.tex
rapport.toc
server
server.c
server.h
server.o
tell_wait.c
tell_wait.h
tell_wait.o
test
\end{lstlisting}
    \chapter{Plusieurs demandes de fichiers par connexions}
      \section{Objectif de la r\'ealisation}
	Dans cette partie, nous avons modifi\'e le serveur afin qu'il g\`ere differentes requ\^etes d'un client sans se d\'econnecter entre chacun d'elle. En d'autres termes il peut ainsi les g\`erer les unes apr\'es les autres sans avoir a se reconnecter. La deconnexion se fait via le client avec une commande "bye".
      \section{Code source comment\'e}
      \section{Tests}

    \chapter{Commandes ls, pwd et cd}
      \section{Objectif de la r\'ealisation}
	Dans cette partie, nous avons ajout\'e les commandes ls (permettant d'afficher le contenu du r\'epertoir courant du serveur FTP), pwd (permettant d'afficher le chemin courant) et cd (permettant de changer le r\'epertoir courant du serveur FTP). \\
      \section{Code source comment\'e}
      \section{Tests}

    \chapter{Commande mkdir, rm, rm -r et put}
      \section{Objectif de la r\'ealisation}
	Dans cette partie, nous avons ajout\'e d'autres commandes concernant la gestion des fichiers pour enrichir notre serveur FTP. Ces commandes sont les commandes mkdir, rm, rm-r et put permettant respectivement de cr\'eer un dossier sur le serveur, supprimer un fichier sur le serveur, supprimer un dossier sur le serveur et t\'el\'everser un fichier). Ces modifications se repercutent sur les autres serveurs esclaves si le client fait ces requ\^etes.
      \section{Code source comment\'e}
      \section{Tests}

    \chapter{Autentifcation}
      \section{Objectif de la r\'ealisation}
	Dans cette partie, nous avons mis en place un systeme d'authentification afin de prot\'eger les fichiers g\`er\'es par le serveur. Pour cela, nous avons mis en place un mot de passe et un login, faisant echouer la connexion au serveur si erron\'es.
      \section{Code source comment\'e}
      \section{Tests}



\end{document}
