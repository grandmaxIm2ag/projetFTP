\documentclass{report}
\usepackage[latin1]{inputenc}
\usepackage[T1]{fontenc}
\usepackage[francais]{babel}
\usepackage{lmodern}
\usepackage{mathrsfs}
\usepackage{mathrsfs}
\usepackage{textcomp}
\usepackage{listings}
\usepackage{siunitx}
\usepackage{tikz}

\usepackage{color}

\definecolor{mygreen}{rgb}{0,0.6,0}
\definecolor{mygray}{rgb}{0.5,0.5,0.5}
\definecolor{mymauve}{rgb}{0.58,0,0.82}

\lstset{ %
  backgroundcolor=\color{white},   % choose the background color; you must add \usepackage{color} or \usepackage{xcolor}; should come as last argument
  basicstyle=\footnotesize,        % the size of the fonts that are used for the code
  breakatwhitespace=false,         % sets if automatic breaks should only happen at whitespace
  breaklines=true,                 % sets automatic line breaking
  captionpos=b,                    % sets the caption-position to bottom
  commentstyle=\color{mygreen},    % comment style
  deletekeywords={...},            % if you want to delete keywords from the given language
  escapeinside={\%*}{*)},          % if you want to add LaTeX within your code
  extendedchars=true,              % lets you use non-ASCII characters; for 8-bits encodings only, does not work with UTF-8
  frame=single,	                   % adds a frame around the code
  keepspaces=true,                 % keeps spaces in text, useful for keeping indentation of code (possibly needs columns=flexible)
  keywordstyle=\color{blue},       % keyword style
  language=C,                      % the language of the code
  morekeywords={*,...},           % if you want to add more keywords to the set
  numbers=left,                    % where to put the line-numbers; possible values are (none, left, right)
  numbersep=5pt,                   % how far the line-numbers are from the code
  numberstyle=\tiny\color{mygray}, % the style that is used for the line-numbers
  rulecolor=\color{black},         % if not set, the frame-color may be changed on line-breaks within not-black text (e.g. comments (green here))
  showspaces=false,                % show spaces everywhere adding particular underscores; it overrides 'showstringspaces'
  showstringspaces=false,          % underline spaces within strings only
  showtabs=false,                  % show tabs within strings adding particular underscores
  stepnumber=2,                    % the step between two line-numbers. If it's 1, each line will be numbered
  stringstyle=\color{mymauve},     % string literal style
  tabsize=2,	                   % sets default tabsize to 2 spaces
  title=\lstname                   % show the filename of files included with \lstinputlisting; also try caption instead of title
}
\oddsidemargin=-0,8cm
\headsep=-1,5cm
\textwidth 18cm
\textheight 25,3cm

\newenvironment{introduction}{%
     \par\noindent\textbf{Remarques}:\noindent
}{
     \par\noindent
}

\author{Grand Maxence, Muller Lucie (Groupe 1) }
\title{Syst\`emes et R\'eseaux : Impl\'ementation d'un pseudo serveur FTP }
\date{09/04/2017}

\begin{document}

\maketitle
\tableofcontents

\chapter*{Introduction}
%Intro projet
	Durant ce projet, nous avons du impl\'ementer un serveur de fichier, inspir\'e des serveurs FTP. Nous commenc\'e avec un serveur dot\'e
  d'une fonction simple : (la r\'ecuperation de fichiers \`a distance), puis nous aons implanter un serveur permettant des actions plus complexes et,
  de ce fait, plus performant.\\
  Pour ce faire nous avons cr\'eer une structure Request permettant de r\'ecu\'erer toutes les informations utiles :
  \begin{lstlisting}
  struct Request{
  	char * cmd; //La methode
  	char * filename; //Le fichier a prendre en commpte
  	char * content; //informtion supplementaire
  	int connfd; //descripteur de socket
  	int fini; //1 si bye envoye, 0 sinon
  	struct stat sbuf;
  };
  \end{lstlisting}
    \chapter{Serveur FTP concurrent}
      \section{Objectif de la r\'ealisation}
	Dans cette partie, nous avons du impl\'ementer la fonction simple du serveur : la r\'ecuperation de fichier \`a distance en donnant le nom du fichier \'a partir du client. Le serveur cr\'e\'e un nouveau fichier sur lequel, \`a l'aide d'un buffer, il copie le contenu du premier fichier. Ce nouveau fichier est sauvegard\'e a l'emplacement courant du client, sous le nom "nomdupremier fichier1"\\
      \section{Code source comment\'e}
      \subsection{client.c}
      \begin{lstlisting}
int main(int argc, char **argv)
{
  int clientfd, fd;
  char *buf, *host, *hidefile;
  struct timeval start, end;
  size_t len=0, n=0;

  if (argc != 2) {
    fprintf(stderr, "usage: %s <host>\n", argv[0]);
    exit(0);
  }
  host = argv[1];//On recupere le hostname du serveur (127.0.0.1 pour les tests)

  clientfd = Open_clientfd(host, port);//On demande la connexion au serveur

  req = malloc(sizeof(struct Request));//On alloue de la memoire pour la Request
  req->clientfd = clientfd;//On recupere le descripteur du socket

  Rio_readinitb(&rio, clientfd);//On initialise Rio

  printf("client connected to server %s\n", host);

  while (1) {
    printf("Tapez votre requete : \n");
    buf = (char*)malloc(sizeof(char)*MAXLINE);
    Fgets(buf, MAXLINE, stdin);
    req->filename = (char*)malloc(sizeof(char)*MAXLINE);
    req->cmd = (char*)malloc(sizeof(char)*MAXLINE);
    req->content = (char*)malloc(sizeof(char)*MAXLINE);
    sscanf(buf, "%s %s %s", req->cmd, req->filename, req->content);//On recupere la requette tapee par le client

    if(!strcmp(req->cmd, "get")){
      gettimeofday(&start, NULL);
      Rio_writen(req->clientfd, buf, strlen(buf));//On envoie la requete au serveur
      fd = open(strcat(req->filename, "1"), O_RDWR | O_CREAT, 0666);//On creer un fichier pour le téléchargement
      free (buf);
      buf = (char*)malloc(sizeof(char)*MAXSEND);
      while((n=Rio_readn(req->clientfd, buf, MAXSEND)) > 0) {//Tant qu'on recoit des donnees, on les ecrit dans le fichier
        rio_writen(fd, buf, n);
        len+=n;
        free(buf);
        buf = (char*) malloc(sizeof(char)*MAXSEND);
      }
      Close(fd);
      gettimeofday(&end, NULL);
      double temps = ((end.tv_sec+(double)end.tv_usec/1000000)-(start.tv_sec+(double)start.tv_usec/1000000));
      printf("%lu bytes received in %f sec (%f bytes / sec) \n",len, temps, ((double)(len/temps)));//On affiche les statistiques
      if(len == 0){
        printf("Le téléchargement a echoée\n");
        remove(req->filename);
      }
    }else if(!strcmp(req->cmd, "bye")){//On ferme la connexion
      Rio_writen(req->clientfd, req->cmd, strlen(req->cmd));
      printf("fin de la connexion\n");
      exit(0);
    }else{//Si la commande est inconnue on ferme la connexion
      printf("Commande %s inconnue\n", req->cmd);
      printf("fin de la connexion\n");
      exit(0);
    }
  }
}
      \end{lstlisting}
      \subsection{server.c}
      \begin{lstlisting}
      int main(int argc, char **argv)
      {
      	Signal(SIGINT, stop);
      	Signal(SIGCHLD, handler);
          pid_t p;
          int listenfd, connfd;
          socklen_t clientlen;
          struct sockaddr_in clientaddr;
          char client_ip_string[INET_ADDRSTRLEN];
          char client_hostname[MAX_NAME_LEN];

          clientlen = (socklen_t)sizeof(clientaddr);

          listenfd = Open_listenfd(port);
          for(int i=0; i<NB_PROC; i++){
              if((p=Fork())==0){
                  while (1) {
                      if((connfd = Accept(listenfd, (SA *)&clientaddr, &clientlen))>=0){
      									//On accept la connexion entrante

                          /* determine the name of the client */
                          Getnameinfo((SA *) &clientaddr, clientlen,client_hostname, MAX_NAME_LEN, 0, 0, 0);

                          /* determine the textual representation of the client's IP address */
                          Inet_ntop(AF_INET, &clientaddr.sin_addr, client_ip_string,INET_ADDRSTRLEN);


                          printf("server connected to %s (%s) %d\n", client_hostname,client_ip_string, getpid());
                          req = malloc(sizeof(struct Request));
                          req->connfd = connfd;

                          readRequest(req);//On lit la requete du client
      										Close(connfd);
                      }
                  }
          				freeRequest(req);
                  exit(0);
              }
              child[i] = p;
          }

          for(int i=0; i<NB_PROC; i++)
              waitpid(child[i], NULL, 0);
      		//On attend tous les fils

          exit(0);
      }

      void handler(int sig){
      	//On s'occupe des zombis
          waitpid(-1, NULL, WNOHANG|WUNTRACED);
          return;
      }
      void stop(int sig){
      	//Si on recoit un SIGCTR, on tue tous les fils
          for(int i=0; i<NB_PROC; i++){
              kill(SIGKILL, child[i]);
          }
          exit(0);
      }

      void readRequest(struct Request *req){

          size_t n;
          req->cmd = (char*)malloc(sizeof(char)*MAXLINE);
          req->filename = (char*)malloc(sizeof(char)*MAXLINE);
          req->content = (char*)malloc(sizeof(char)*MAXLINE);
          char* request = (char *)malloc(sizeof(char)*MAXLINE);
          rio_t rio;
          Rio_readinitb(&rio, req->connfd);
          n=Rio_readlineb(&rio, request, MAXLINE);
          fflush(stdout);
          request[n-1] = '\0';

          sscanf(request, "%s %s %s", req->cmd, req->filename, req->content);
      		//On recupere la requte , et remplis notre Request
      		stat(req->filename, &req->sbuf);
          if(!strcmp("get", req->cmd)){
          	get(req);
          }else if (!strcmp("bye", req->cmd) ) {
          	printf("fin de la connexion");
      			Close(req->connfd);
          }
          fflush(stdout);
      }

      void get(struct Request *req){
      	char buf[MAXSEND];
      	size_t n,err;
      	rio_t rio;


      		//On envoie d'un coup
      		int srcfd;
          char *srcp;
      		//Rio_readinitb(&rio, fd);
          srcfd = Open(req->filename, O_RDONLY, 0);
          srcp = Mmap(0, req->sbuf.st_size, PROT_READ, MAP_PRIVATE, srcfd, 0);
          Close(srcfd);
          Rio_writen(req->connfd, srcp, req->sbuf.st_size);
          Munmap(srcp, req->sbuf.st_size);
      }
      \end{lstlisting}
      \section{Tests}
\begin{lstlisting}[frame=single,basicstyle=\footnotesize,language=bash]
maxence@Sybil:~$ cat Makefile
CC=gcc
CFLAGS=-Wall
LIBS=-lpthread
all: server client
server: server.o csapp.o tell_wait.o
$(CC) $(CFLAGS) -o server server.o csapp.o tell_wait.o  $(LIBS)

client: client.o csapp.o
$(CC) $(CFLAGS) -o client  client.o csapp.o $(LIBS)
client.o: client.c csapp.c client.h csapp.h
$(CC) $(CFLAGS) -c client.c csapp.c $(LIBS)

server.o: server.c csapp.c server.h csapp.h
		$(CC) $(CFLAGS) -c server.c csapp.c $(LIBS)
tell_wait.o: tell_wait.c csapp.c tell_wait.h csapp.h
	$(CC) $(CFLAGS) -c tell_wait.c csapp.c $(LIBS)

csapp.o: csapp.c csapp.h
		$(CC) $(CFLAGS) -c csapp.c $(LIBS)
clean:
  	rm server client *.o
maxence@Sybil:~$ ls
backupServer  client.c  client.o  csapp.h  Makefile  server    server.h  tell_wait.c  tell_wait.o  test_pipe.c
client        client.h  csapp.c   csapp.o  master.c  server.c  server.o  tell_wait.h  test
maxence@Sybil:~$ ./client 127.0.0.1
client connected to server 127.0.0.1
Tapez votre requete :
get Makefile
617 bytes received in 0.002723 sec (226590.103143 bytes / sec)
Tapez votre requete :
bye
fin de la connexion
maxence@Sybil:~$ cat Makefile1
CC=gcc
CFLAGS=-Wall
LIBS=-lpthread

all: server client


server: server.o csapp.o tell_wait.o
  $(CC) $(CFLAGS) -o server server.o csapp.o tell_wait.o  $(LIBS)
client: client.o csapp.o
	$(CC) $(CFLAGS) -o client  client.o csapp.o $(LIBS)

client.o: client.c csapp.c client.h csapp.h
	$(CC) $(CFLAGS) -c client.c csapp.c $(LIBS)
server.o: server.c csapp.c server.h csapp.h
	$(CC) $(CFLAGS) -c server.c csapp.c $(LIBS)

tell_wait.o: tell_wait.c csapp.c tell_wait.h csapp.h
	$(CC) $(CFLAGS) -c tell_wait.c csapp.c $(LIBS)
csapp.o: csapp.c csapp.h
	$(CC) $(CFLAGS) -c csapp.c $(LIBS)

clean:
	rm server client *.o
\end{lstlisting}

    \chapter{D\'ecoupage du fichier}
      \section{Objectif de la r\'ealisation}
	Dans cette partie, nous avons du ameliorer le serveur afin de ne pas envoyer le fichier en une seule fois, mais en plusieurs paquets. Cela permet de ne pas surcharger la m\'emoire si un fichier de grande taille est transfer\'e. Pour cela on utilise une boucle qui envoie a chaque fois MAXSEND (maximum d'octets envoy\'es) octets au client.\\
      \section{Code source comment\'e}
\subsection{server.c}
Seul la fonction get de server.c a \'et\'e modifi\'ee

  \begin{lstlisting}
void get(struct equest *req){
  	char buf[MAXSEND];
  	size_t n,err;
  	rio_t rio;

  	int fd = open(req->filename, O_RDONLY);
  	Rio_readinitb(&rio, fd);
  	/*
  Tant qu'on a ecris, on continue d'ecrire
  	*/
  	while ((n = Rio_readlineb(&rio, buf, MAXSEND)) > 0) {
  			err = rio_writen(req->connfd, buf, n);
  	    if(err == -1){//On stoppe si le client s'arrete
  		     	fprintf(stderr, "Arret innatendu du client\n");
  		     	break;
  	    }
  	}

  }
  \end{lstlisting}
      \section{Tests}
\begin{lstlisting}[frame=single,basicstyle=\footnotesize,language=bash]
maxence@Sybil:~$ cat test
backupServer
client
client.c
client.h
client.o
csapp.c
csapp.h
csapp.o
Makefile
Makefile1
master.c
server
server.c
server.h
server.o
tell_wait.c
tell_wait.h
tell_wait.o
test
test_pipe.c
maxence@Sybil:~$ ./client 127.0.0.1
client connected to server 127.0.0.1
Tapez votre requete :
get test
186 bytes received in 0.009943 sec (18706.611932 bytes / sec)
Tapez votre requete :
bye
fin de la connexion
maxence@Sybil:~$ cat test1
backupServer
client
client.c
client.h
client.o
csapp.c
csapp.h
csapp.o
Makefile
Makefile1
master.c
server
server.c
server.h
server.o
tell_wait.c
tell_wait.h
tell_wait.o
test
test_pipe.c
\end{lstlisting}
    \chapter{Gestion simple des pannes cot\'e client}
      \section{Objectif de la r\'ealisation}
	Dans cette partie, nous devons g\`erer les pannes du client, c'est-a-dire le comportement du serveur au cas o\`u le client s'arr\^eterais
  durant le transfert.
  Un fichier temporaire est cr\'e\'e au d\'ebut du processus et lorsque ce dernier est termin\'e le fichier est supprim\'e.
  Il n'est pas supprim\'e si le client s'arr\^ete avant la fin du t\'el\'echargement. Le client v\'erifie, a l'appel du transfert,
  si un tel fichier temporaire existe. Si c'est le cas cela signifie que le serveur a d\'ej\`a commenc\'e un transfert
  mais que celui ci a \'etait arr\^et\'e avant sa terminaison.
  Le client r\'ecup\`ere alors ce fichier temporaire et sait quelle quantit\'e de donn\'ee il lui reste \`a t\'el\'echarger. \\
      \section{Code source comment\'e}
      Seul le fichier server.c a \'et\'e modifi\'e
            \subsection{client.c}
            \begin{lstlisting}
if(!strcmp(req->cmd, "get")){
  gettimeofday(&start, NULL);
  Rio_writen(req->clientfd, buf, strlen(buf));//On envoie la requete au serveur
  hidefile = (char*)malloc(sizeof(char)*MAX_NAME_LEN);
  hidefile[0]='T';
  strcat(hidefile, req->filename);//On creer le nom du fichier temporaire
  if(stat(hidefile,&req->sbuf) != 0){//Si le fichier temporaire n'existe pas
    open(hidefile, O_RDWR | O_CREAT, 0666);//On le cree
    fd = open(strcat(req->filename, "1"), O_RDWR | O_CREAT, 0666);//On cree un fichier pour le téléchargement
    free (buf);
    buf = (char*)malloc(sizeof(char)*MAXSEND);
    while((n=Rio_readn(req->clientfd, buf, MAXSEND)) > 0) {//Tant qu'on recoit des donnees, on les ecrit dans le fichier
      rio_writen(fd, buf, n);
      len+=n;
      free(buf);
      buf = (char*) malloc(sizeof(char)*MAXSEND);
    }
    Close(fd);
    gettimeofday(&end, NULL);
    double temps = ((end.tv_sec+(double)end.tv_usec/1000000)-(start.tv_sec+(double)start.tv_usec/1000000));
    printf("%lu bytes received in %f sec (%f bytes / sec) \n",len, temps, ((double)(len/temps)));//On affiche les statistiques
      if(len == 0){
      printf("Le téléchargement a echoue\n");
      remove(req->filename);
    }
    remove(hidefile);
  }else{//Si le fichier temporaire existe
    fd = open(strcat(req->filename, "1"), O_WRONLY);
    stat(req->filename,&req->sbuf);
    int dejaLu = req->sbuf.st_size;//On recupere la taille des donnees deja telechargees
    free (buf);
    buf = (char*)malloc(sizeof(char)*MAXSEND);
    while((n=Rio_readn(req->clientfd, buf, MAXSEND)) > 0) {
      int tmp = dejaLu;
      dejaLu-=n;
      if(dejaLu <= 0){//On ecrit si ce sont des donnees non téléchargées
      rio_writen(fd, buf, n);
      free(buf);
      buf = (char*) malloc(sizeof(char)*MAXSEND);
      len+=n;
    }else{//Sinon on déplace le curseur
      if(n<tmp)
          lseek(fd, n, SEEK_CUR);
      else
          lseek(fd, n-tmp, SEEK_CUR);
      }
    }
    gettimeofday(&end, NULL);
    double temps = ((end.tv_sec+(double)end.tv_usec/1000000)-(start.tv_sec+(double)start.tv_usec/1000000));
      printf("%lu bytes received in %f sec (%f bytes / sec) \n",len, temps, ((double)(len/temps)));
      remove(hidefile);
    }
  }else if(!strcmp(req->cmd, "bye")){//On ferme la connexion
    Rio_writen(req->clientfd, req->cmd, strlen(req->cmd));
    printf("fin de la connexion\n");
    exit(0);
  }else{//On stop la connexion
    printf("Commande %s inconnue\n", req->cmd);
    printf("fin de la connexion\n");
    exit(0);
  }
}
            \end{lstlisting}
      \section{Tests}
      \begin{lstlisting}[frame=single,basicstyle=\footnotesize,language=bash]
maxence@Sybil:~$ rm test1
maxence@Sybil:~$ head -3 test > test1
maxence@Sybil:~$ touch Ttest
maxence@Sybil:~$ ./client 127.0.0.1
client connected to server 127.0.0.1
Tapez votre requete :
get test
186 bytes received in 0.000816 sec (227977.949737 bytes / sec)
Tapez votre requete :
bye
fin de la connexion
maxence@Sybil:~$ cat test1
backupServer
client
client.c
client.h
client.o
csapp.c
csapp.h
csapp.o
Makefile
Makefile1
master.c
server
server.c
server.h
server.o
tell_wait.c
tell_wait.h
tell_wait.o
test
\end{lstlisting}

    \chapter{Serveur FTP avec \'equilibrage des charges}
      \section{Objectif de la r\'ealisation}
	Dans cette partie, nous avons ajout\'e un r\'epartisseur de charge. Il sert a partager les differentes requ\^etes des clients entre
  plusieurs serveurs dit "esclaves". Cette r\'epartition est g\`er\'ee par un serveur "maitre" qui re\c coit toutes les requ\^etes
  avant de les envoyer aux esclaves. Lorsque le serveur ma\^itre re\c coit une connexion vennant d'un client, il lui ernvoie le port d'un esclave,
  auquel le client doit se connecter.
      \section{Code source comment\'e}
      \subsection{master.c}
      \begin{lstlisting}
int sCourant = 0;

int slave[NB_PROC];

int next(){//Renvoie le prochain esclave
	int p = slave[sCourant];
	sCourant = ++sCourant % NB_PROC;
	return p;
}
int main(int argc, char **argv)
{
  int listenfd, connfd;
  socklen_t clientlen;
  struct sockaddr_in clientaddr;
  char client_ip_string[INET_ADDRSTRLEN];
  char client_hostname[MAX_NAME_LEN];
	rio_t rio;

  clientlen = (socklen_t)sizeof(clientaddr);

  listenfd = Open_listenfd(port);

  for(int i =0; i<NB_PROC; i++){
    slave[i] = port+i+1;
  }
  while(1){
    if((connfd = Accept(listenfd, (SA *)&clientaddr, &clientlen))>=0){

      /* determine the name of the client */
      Getnameinfo((SA *) &clientaddr, clientlen,client_hostname, MAX_NAME_LEN, 0, 0, 0);

      /* determine the textual representation of the client's IP address */
      Inet_ntop(AF_INET, &clientaddr.sin_addr, client_ip_string,INET_ADDRSTRLEN);


      printf("server connected to %s (%s) %d\n", client_hostname,client_ip_string, getpid());

  		Rio_readinitb(&rio, connfd);
  		int p = next();
  		Rio_writen(connfd, &p, sizeof(p));//Envoie au client le numero de port de l'esclve traitant

      Close(connfd);//Ferme la connexion avec le client
    }
  }
  exit(0);
}
      \end{lstlisting}
      \subsection{server.c}
      \begin{lstlisting}
clientfd = Open_clientfd(host, port);//On demande la connexion au serveur maitre
printf("client connected to server %s\n", host);

Rio_readn(clientfd, &port2, sizeof(int)); //On recupert le port du serveur esclave

printf("Vous aller être redirigé vers l'esclave %d\n", port2);

clientfd = Open_clientfd(host, port2);//On demande la connexion au serveur esclave

req = malloc(sizeof(struct Request));//On alloue de la memoire pour la Request
req->clientfd = clientfd;//On recupere le descripteur du socket
      \end{lstlisting}
      \section{Tests}
\begin{lstlisting}[frame=single,basicstyle=\footnotesize,language=bash]
maxence@Sybil:~$ ls > test
maxence@Sybil:~$ ./client 127.0.0.1
client connected to server 127.0.0.1
Vous aller être redirigé vers l'esclave 2122
Tapez votre requete :
get test
238 bytes received in 0.004584 sec (51921.582857 bytes / sec)
Tapez votre requete :
bye
fin de la connexion
maxence@Sybil:~$ cat test
backupServer
client
client.c
client.h
client.o
csapp.c
csapp.h
csapp.o
debug
Makefile
master
master.c
master.o
projetFTPBackup
rapport.pdf
rapport.tex
rapport.toc
server
server.c
server.h
server.o
tell_wait.c
tell_wait.h
tell_wait.o
test
\end{lstlisting}
    \chapter{Plusieurs demandes de fichiers par connexions}
      \section{Objectif de la r\'ealisation}
	Dans cette partie, nous avons modifi\'e le serveur afin qu'il g\`ere differentes requ\^etes d'un client sans fermer la connexion
  entre chacun d'elle. En d'autres termes il peut ainsi les g\`erer les unes apr\'es les autres sans avoir a se reconnecter.
  La fermeture de connexion se fait via le client avec une commande "bye".
      \section{Code source comment\'e}
      \subsection{server.c}
      Tout d'aboord on rejoute l'option suivante dans la fonction readRequest :
      \begin{lstlisting}
}else if (!strcmp("bye", req->cmd) ) {//Si le client envoie bye, on ferme la connexion
	printf("fin de la connexion\n");
  	Close(req->connfd);
  	req->fini=1;
  }
      \end{lstlisting}
      Et pour le main, maintenant on lit les requ\^etes ennvoy\'ees tant que l'on a pas re\c cue bye :
      \begin{lstlisting}
while(!req->fini){
  readRequest(req, 0); //On lit la requete du client
}
      \end{lstlisting}
      \section{Tests}
\begin{lstlisting}[frame=single,basicstyle=\footnotesize,language=bash]
maxence@Sybil:~$ ls > test
maxence@Sybil:~$ ls -l > testt
maxence@Sybil:~$ ./client 127.0.0.1
client connected to server 127.0.0.1
Vous allez être redirigé vers l'esclave 2122
Tapez votre requete :
get test
262 bytes received in 5.005285 sec (52.344671 bytes / sec)
Tapez votre requete :
get testt
1969 bytes received in 5.008208 sec (393.154595 bytes / sec)
Tapez votre requete :
bye
fin de la connexion
maxence@Sybil:~$ cat test1
backupServer
client
client.c
client.c1
client.h
client.o
csapp.c
csapp.h
csapp.o
debug
Makefile
master
master.c
master.o
projetFTPBackup
rapport.aux
rapport.log
rapport.pdf
rapport.tex
rapport.toc
server
server.c
server.h
server.o
t
tell_wait.c
tell_wait.h
test
maxence@Sybil:~$ cat testt1
total 3456
-rwx------ 1 maxence maxence   3294 avril  1 11:41 backupServer
-rwx------ 1 maxence maxence  36488 avril  9 18:29 client
-rwx------ 1 maxence maxence   7025 avril  9 18:21 client.c
-rwx------ 1 maxence maxence   7018 avril  9 16:23 client.c1
-rwx------ 1 maxence maxence    224 avril  4 16:05 client.h
-rwx------ 1 maxence maxence   9952 avril  9 18:21 client.o
-rwx------ 1 maxence maxence  20259 mars  29 18:33 csapp.c
-rwx------ 1 maxence maxence   6105 mars  29 18:33 csapp.h
-rwx------ 1 maxence maxence  23760 avril  9 18:29 csapp.o
-rwx------ 1 maxence maxence      0 avril  9 16:45 debug
-rwx------ 1 maxence maxence    960 avril  9 16:24 Makefile
-rwx------ 1 maxence maxence  36704 avril  9 18:29 master
-rwx------ 1 maxence maxence   6867 avril  9 18:27 master.c
-rwx------ 1 maxence maxence  10136 avril  9 18:29 master.o
drwx------ 1 maxence maxence 131072 avril  9 18:14 projetFTPBackup
-rwx------ 1 maxence maxence   4046 avril  9 18:39 rapport.aux
-rwx------ 1 maxence maxence  44067 avril  9 18:39 rapport.log
-rwx------ 1 maxence maxence 259857 avril  9 18:39 rapport.pdf
-rwx------ 1 maxence maxence  22400 avril  9 18:39 rapport.tex
-rwx------ 1 maxence maxence   2538 avril  9 18:39 rapport.toc
-rwx------ 1 maxence maxence  36776 avril  9 18:29 server
-rwx------ 1 maxence maxence   6491 avril  9 18:28 server.c
-rwx------ 1 maxence maxence    452 avril  9 16:04 server.h
-rwx------ 1 maxence maxence  10728 avril  9 18:29 server.o
-rwx------ 1 maxence maxence   1794 févr. 28 09:24 tell_wait.c
-rwx------ 1 maxence maxence    131 févr. 28 08:55 tell_wait.h
-rwx------ 1 maxence maxence    262 avril  9 18:42 test
-rwx------ 1 maxence maxence      0 avril  9 18:42 testt
\end{lstlisting}
    \chapter{Commandes ls, pwd et cd}
      \section{Objectif de la r\'ealisation}
	Dans cette partie, nous avons ajout\'e les commandes ls (permettant d'afficher le contenu du r\'epertoire courant du serveur FTP),
  pwd (permettant d'afficher le chemin courant) et cd (permettant de changer le r\'epertoire courant du serveur FTP). Pour les commandes ls et pwd,
  le serveur cr\'eer un procecus fils qui ex\'ecute la commande \`a l'aide de execvp, et pour cd, nous utilisons la fonction chdir.\\
      \section{Code source comment\'e}
      \subsection{server.c}
\begin{lstlisting}
if (!strcmp("ls", req->cmd)){
    char* args[1];
args[0] = req->cmd;
int c;
int fd = open(".log", O_RDWR | O_CREAT, 0666);
if((c=Fork())==0){
    close(1);
    dup(fd);
    execvp(args[0], args);
    exit(0);
}
close(fd);
waitpid(c, NULL, 0);
memcpy(req->filename, ".log", 4);
stat(req->filename, &req->sbuf);
int l = req->sbuf.st_size;
rio_writen(req->connfd, &l, sizeof(int));
    get(req);
    remove(".log");
}else if (!strcmp("pwd", req->cmd)){
  char* args[1];
  args[0] = "pwd";
  int c;
  int fd = open(".log", O_RDWR | O_CREAT, 0666);
  if((c=Fork())==0){
      close(1);
    dup(fd);
    execvp(args[0], args);
    exit(0);
  }
  close(fd);
  waitpid(c, NULL, 0);
  memcpy(req->filename, ".log", 4);
  stat(req->filename, &req->sbuf);
  int l = req->sbuf.st_size;
  rio_writen(req->connfd, &l, sizeof(int));
  get(req);
  remove(".log");
}else if (!strcmp("cd", req->cmd)){
    chdir(req->filename);//On se deplace dans req->filename (qui est ici un repertoire et non pas un fichier)
}
      \end{lstlisting}
      \section{Tests}
      \begin{lstlisting}[frame=single,basicstyle=\footnotesize,language=bash]
maxence@Sybil:~$ ./client 127.0.0.1
client connected to server 127.0.0.1
Vous allez être redirigé vers l'esclave 2122
Tapez votre requete :
ls
backupServer
client
client.c
client.h
client.o
csapp.c
csapp.h
csapp.o
debug
Makefile
master
master.c
master.o
projetFTPBackup
rapport.pdf
rapport.tex
server
server.c
server.h
server.o
tell_wait.c
tell_wait.h
test
testt

Tapez votre requete :
pwd
/media/maxence/MyLinuxLive1/projetFTP
Tapez votre requete :
cd ..
Tapez votre requete :
ls
backup
cours
favoris.html
informatique
latex
Papier
projetFTP
projetFTPBackup
Revisions_S5
Revisions_S6-master
SAGA MP3
System Volume Information
t

Tapez votre requete :
pwd
/media/maxence/MyLinuxLive1

Tapez votre requete :
bye
fin de la connexion

      \end{lstlisting}
    \chapter{Autentifcation}
      \section{Objectif de la r\'ealisation}
	Dans cette partie, nous avons mis en place un syst\`eme d'authentification afin de prot\'eger les fichiers g\`er\'es par le serveur.
  Pour cela, nous avons mis en place un mot de passe et un login, faisant \'echouer la connexion au serveur si erron\'ee. Pour ce faire,
  nous disposons d'un fichier .login dans lequel se trouve \`a chaque ligne : login::password.
      \section{Code source comment\'e}
      \subsection{master.c}
      \begin{lstlisting}
int auth(char * login){
    int b=0;
    char buf[MAXLINE];

  	int fd = open(".login", O_RDONLY);//On ouvre le fichier comportant tous les login::passxord
  	rio_t rio;
  	Rio_readinitb(&rio, fd);

  	size_t n =0;

  	while((n = Rio_readlineb(&rio, buf, MAXLINE))>0 && !b){
  			//On lit chacune des lignes
		//Si on lit le login envoye par le clientfd
		//b prend la valeur 1 et on arrete la lecture
		buf[n-1]='\0';
  		if(!strcmp(buf, login)){
  				b=1;
  				break;
  		}
    }
return b;
}
      \end{lstlisting}
      \section{Tests}
  \begin{lstlisting}[frame=single,basicstyle=\footnotesize,language=bash]
  maxence@Sybil:~$ ./client 127.0.0.1
  client connected to server 127.0.0.1
  login : erreur
  Password : erreur
  L'authentification a echouee
  maxence@Sybil:~$ ./client 127.0.0.1
  client connected to server 127.0.0.1
  login : toto
  Password : toto
  Vous allez être redirigé vers l'esclave 2122
  \end{lstlisting}

  \chapter{Les commandes rm, rm -r, mkdir et put}
  Nous n'avons pas r\'eussi a implanter cette partie. Nous avons rencontr\'e deux probl\`emes : tout d'abord nous avions un probl\`eme
  de bufferisation, lorsque nous utilisons une de ses commandes, si nous souhaitons ensuite utiliser la commande ls ou pwd,
  le serveur garde en m\'emoire le nom du fichier ou repertoire cr\'e\'e ou supprimer, par exemple apr\`es avoir taper : mkdir repTest,
  lorsque le client tape ls, le serveur cherche \`a ex\'ecuter ls repTest. Nous avons aussi un probl\`eme de communication ma\^itre/esclave. N\'eanmoins,
  si nous tapons la commande rm, rm -r ou mkdir, le serveur cr\'er/supprime le fihcier/r\'epertoire concern\'e. Et la m\'ethode put, t\'el\'everse le
  fichier.

  \section{code source comment\'e}
  \subsection{server.c}

  Nous avons rajout\'e les options dans la fonction readRequest :
  \begin{lstlisting}
  else if (!strcmp("put", req->cmd) || !strcmp("rm", req->cmd) || !strcmp("mkdir", req->cmd) ) {
  //Si c'est une des requetes causant une modification des servers
    if(!strcmp("rm", req->cmd)){
      char* args[3];
      if(!strcmp(req->filename, "-r")){
          args[0] = "rm";
          args[1] = "-r";
          args[2] = req->content;
      }else{
          args[0] = "rm";
          args[1] = req->filename;
      }
      int c;
      if((c=Fork())==0){
          execvp(args[0], args);
          exit(0);
      }
      waitpid(c, NULL, 0);
    }if(!strcmp("mkdir", req->cmd)){
      char* args[2];
      args[0] = "mkdir";
      args[1] = req->filename;
      int c;
      if((c=Fork())==0){
          execvp(args[0], args);
          exit(0);
      }
      waitpid(c, NULL, 0);

    }else{
      put(req);
    }
  \end{lstlisting}
  Et nous avons rejouter une fonction put :
  \begin{lstlisting}
  void put(struct Request* r) {
  		int l; ssize_t n=0;
  		rio_readn(req->connfd, &l, sizeof(int));
  		char * buf;
  		int fd = open(strcat(req->filename, "1"), O_RDWR | O_CREAT, 0666);//On creer un fichier pour le téléchargement
  		buf = (char*)malloc(sizeof(char)*MAXSEND);
  		int send=MAXSEND;
  		if(l<MAXSEND)
  			send=l;
  		sleep(5);
  		while((n=rio_readn(req->connfd,buf, send)) > 0 && l>0) {//Tant qu'on recoit des données, on les ecrit dans le fichier
  				if (n>send)
  					n=send;
  				rio_writen(fd, buf, n);
  				l-=n;
  				if(l<MAXSEND)
  					send=l;
  		}
  		close(fd);
  }
  \end{lstlisting}

  \subsection{client.c}

  Nous rajoutons les options suivantes pour client.c :

  \begin{lstlisting}
  else if (!strcmp("put", req->cmd)){
    //Si c'est une des requetes causant une modification des servers
    stat(req->filename, &req->sbuf);
    int l = req->sbuf.st_size;
    rio_writen(req->clientfd, &l, sizeof(int));
    int fd = open(req->filename, O_RDONLY);
    /*
    Tant qu'on a ecris, on continue d'ecrire
    */
    do {
        n = rio_readn(fd, buf, MAXSEND);
        rio_writen(req->clientfd, buf, n);
    }while (n>0);

    close(fd);
  }else if(!strcmp("cd", req->cmd)){}
  else if(!strcmp("rm", req->cmd) || !strcmp("mkdir", req->cmd)){}
  \end{lstlisting}


\end{document}
